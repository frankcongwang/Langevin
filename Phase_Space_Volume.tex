\documentclass{article}
\usepackage{amsmath}
\usepackage{mathrsfs}
\title{Phase Space Volumes and Liouville's Theorem }
\author{Frank}
\begin{document}
\maketitle
\section{Coordinate transformation in phase space for Hamiltonian and non-Hamiltonian system}
Since the mapping function of the point $x_0$ to $x_t$ is one-to-one, the mapping is equivalent to a coordinate transformation on the phase space from initial phase space coordinate to final coordinate
\begin{align}
dx_t= Jdx_0
\end{align}
where $J$ is the Jacobian of the transformation and $J_{kl}=\frac{\partial x^k_t}{\partial x^l_0}$. The determinant of $J$ is named
\begin{align}
J(x_t;x_0)=det(J)
\end{align}
Since $J$ is diagonal, it has eigenvalue $\lambda_k$ and $ln(J)$ has eigenvalue $ln(\lambda)$, so
\begin{align}
e^{Tr[ln(J)]}&=e^{\sum_kln(\lambda_k)}\nonumber\\
&=\prod_k\lambda_k\nonumber\\
&=det(J)
\end{align}
Then
\begin{align}
\frac{d}{dt}J(x_t;x_0)&=\frac{d}{dt}e^{Tr[ln(J)]}\nonumber\\
&=e^{Tr[ln(J)]}Tr\bigg[\frac{dJ}{dt}J^{-1}\bigg]\nonumber\\
&=J(x_t;x_0)\sum_{k,l}\bigg[\frac{dJ_{kl}}{dt}J^{-1}_{lk}\bigg]\nonumber\\
&=J(x_t;x_0)\sum_{k,l}\bigg[\frac{\partial\dot{x}^k_t}{\partial x^l_0}\frac{\partial x^l_0}{\partial x^k_0}\bigg]\nonumber\\
&=J(x_t;x_0)\sum_k\frac{\partial\dot{x}^k_t}{\partial x^k_0}
\end{align}
For a system evolving under Hamilton equation, the phase space compressibility $\kappa(x_t,t)=\nabla\cdot x_t=\sum_k\frac{\partial\dot{x}^k_t}{\partial x^k_0}=0$. the equation of motion for the Jacobian reduced to
\begin{align}
\frac{d}{dt}J(x_t;x_0)=0
\end{align}
This equation implies that the Jacobian is a constant at all time. Since the initial value of $J(x_t;x_0)$ is 1, it remains 1 at all time. It implies that the phase space volume is a constant, which is known as the Liouville's theorem.\\
For a non-Hamilton system, if there's a function $\omega(x_t,t)$ such that $\kappa(x_t,t)=\frac{d}{dt}\omega(x_t,t)$,
\begin{align}
J(x_t;x_0)=\text{exp}[\omega(x_t,t)-\omega(x_0,0)]
\end{align}
and the phase space volume element evolve according to
\begin{align}
\text{exp}[-\omega(x_t,t)]dx_t=\text{exp}[-\omega(x_0,0)]dx_0
\end{align}
The equation constitutes a generalized Liouville theorem which implies a weighted phase space volume is conserved. So the phase space does not follow the usual laws of Euclidean geometry. The phase space volume can be denoted as $\sqrt{g(x)}dx$, where $g(x)$ is the determinant of a second-rank tensor $g_{ij}(x)$ known as the \textbf{metric tensor}.\\
The Jacobian can be as a statement of the fact of the coordinate transformation $x_0\rightarrow x_t$
\begin{align}
J(x_t;x_0)=\frac{\sqrt{g(x_0,0)}}{\sqrt{g(x_t,t)}}
\end{align}
where
\begin{align}
\sqrt{g(x_t,t)}=e^{-\omega{x_t,t}}
\end{align}
The implication of the equation is that any phase space integral that represents an ensemble average should be performed using $\sqrt{g}$ as the volume element.
\section{Generalization of the Liouville equation in non-Hamiltonian system}
Assume a system interacting with its surroundings and possibly subject to driving force is described bt non-Hamiltonian microscopic equation of the form
\begin{align}
\dot{x}=\xi(x,t)
\end{align}
Consider an ensemble described by a distribution function $f:R^{n+1}\rightarrow R^1$, which is a function of $n$ coordinate and time $t$. A continuity equation for $f$ can be derived whose condition is that the rate if change of the number of ensemble members within volume $\Omega$ is balanced by the flux of members through the surface bounding $\Omega$, which is expressed mathematically as
\begin{align}
-\frac{d}{dt}\int_{\Omega}f(x,t)\sqrt{g(x,t)}dx&=\int_{\partial\Omega}\tilde{\sigma}\hat{n}\cdot \xi f\nonumber\\
&=\int_{\Omega}\mathcal{L}_{\xi}(fg(x,t)dx)
\end{align}
where $\tilde{\sigma}$ is the surface $n-1$ form and $\hat{n}$ is the unit normal one-form to the surface. The surface integral has been converted to a volume integral via a generalization of the divergence theorem to manifolds with nontrivial metrics using Lie derivative $\mathcal{L}_{\xi}$ along the vector $\xi$.\\
The equation hold independent of the choice of $\Omega$ and thus implies the local continuity condition
\begin{align}
\int_{\Omega}\bigg(\frac{\partial}{\partial t}+\mathcal{L}_{\xi}\bigg)(f(x,t)\sqrt{g(x,t)}dx)=0
\end{align}
To project the equation onto a coordinate basis, we first apply the Leibniz rule, $\mathcal{L}_{\xi}(f\sqrt{g}dx)=\sqrt{g}dx(\mathcal{L}_{\xi}f)+f\mathcal{L}_{\xi}(\sqrt{g}dx$, to the Lie derivative on the product. The action of the Lie derivative on the scalar and on the volume form is
\begin{align}
\mathcal{L}_{\xi}f&=\xi^i\frac{\partial f}{\partial x^i}\nonumber\\
\mathcal{L}_{\xi}\epsilon_{i_1\cdots\i_n}&=\dot{x}^k\frac{\partial\sqrt{g}}{\partial x^k}\epsilon_{i_1\cdots\i_n}+\sqrt{g}\frac{\partial\xi^i}{\partial x^k}\epsilon_{i_1\cdots\i_n}
\end{align}
where the component representation of the wedge product is given by $\epsilon_{i_1\cdots\i_n}$, the Levi-Civita tensor.\\
Combining he last two equations gives the general form for the continuity equation in an arbitrary coordinate basis
\begin{align}
\bigg[\frac{\partial}{\partial t}(f(x,t)\sqrt{g(x,t)})+\frac{\partial\xi^i}{\partial x^k}\bigg]dx&=0\\
\frac{\partial}{\partial t}(f(x,t)\sqrt{g(x,t)})+\nabla\cdot(f(x,t)\sqrt{g(x,t)})&=0
\end{align}
According to equation (4) and (8), the phase space metric factor $\sqrt{g(x,t)}$ satisfies
\begin{align}
\frac{d}{dt}\sqrt{g(x,t)}=-\kappa(x,t)\sqrt{g(x,t)}
\end{align}
The last two equations lead to an equation for $f(x,t)$ alone
\begin{align}
\frac{\partial}{\partial t}f(x,t)+\xi(x,t)\cdot\nabla f(x,t)&=0\\
\frac{d}{dt}f(x,t)&=0
\end{align}
In equilibrium, both $f(x_t.t)$ and $g(x_t,t)$ have no explicit time dependence.\\
According to equation (7), we can obtain the generalized \textbf{Liouville equation}
\begin{align}
f(x_t,t)\sqrt{g(x_t,t)}dx_t=f(x_0,t)\sqrt{g(x_0,t)}dx_0
\end{align}
Suppose the dynamical equations processes a set of $n_c$ associated conservation laws or conserved quantities $\Lambda_k(x),k=1,\cdots,n_c$, which satisfies
\begin{align*}
\Lambda_k(x_t)-C_k&=0\\
\frac{d\Lambda_k}{dt}&=0
\end{align*}
a general solution for $f(x)$ can be constructed in the form
\begin{align}
f(x)=\prod^{n_c}_{k=1}\delta(\Lambda_k(x)-C_k)
\end{align}
\section{Analysis of Different Algorithms}
The microcanonical ensemble average of the function $F$ is denoted $F_{NVE}$, which is defined as
\begin{align}
F_{NVE}(N,V,E)\equiv [N!\Omega(N,V,E)]^{-1}\int_Vd\textbf{r}^N\int d\textbf{p}^N\delta[\mathscr{H}(\textbf{r}^N,\textbf{p}^N;V)-E]F(\textbf{r}^N,\textbf{p}^N;V)
\end{align}
where 
\begin{align}
\Omega(N,V,E)=(N!)^{-1}\int_Vd\textbf{r}^N\int d\textbf{p}^N\delta[\mathscr{H}(\textbf{r}^N,\textbf{p}^N;V)-E]
\end{align}
is the microcanonical ensemble partition function.\\
The canonical ensemble partition function is
\begin{align}
Q(N,V,T)=(N!)^{-1}\int_Vd\textbf{r}^N\int d\textbf{p}^N \text{exp}\big[-\frac{\mathscr{H}(\textbf{r}^N,\textbf{p}^N;V)}{kT}\big]
\end{align}
The isothermal-isobaric ensemble partition function is
\begin{align}
\Delta(N,P,T)=(N!)^{-1}\int^{\infty}_0dV\int_Vd\textbf{r}^N\int d\textbf{p}^N \text{exp}\big[-\frac{PV+\mathscr{H}(\textbf{r}^N,\textbf{p}^N;V)}{kT}\big]
\end{align}
The isoenthalpic-isobaric ensemble partition function is
\begin{align}
\Gamma(N,P,H)=(N!)^{-1}\int^{\infty}_0 dV\int_Vd\textbf{r}^N\int d\textbf{p}^N\delta[\mathscr{H}(\textbf{r}^N,\textbf{p}^N;V)+PV-H]
\end{align}
\subsection{The Hoover Algorithm}
Hoover introduced the equations of motion as follows
\begin{align}
\dot{r}_i&=\frac{p_i}{m_i}+\frac{p_{\epsilon}}{W}r_i\nonumber\\ 
\dot{p}_i&=F_i-\frac{p_{\epsilon}}{W}p_i-\frac{p_{\xi}}{Q}p_i\nonumber\\
\dot{V}&=\frac{Dp_{\epsilon}}{W}V\nonumber\\
\dot{p_{\epsilon}}&=dV(P-P_{ext})-\frac{p_{\xi}}{Q}p_{\epsilon}\\
\dot{\xi}&=\frac{p_{\xi}}{Q}\nonumber\\
\dot{p_{\xi}}&=\sum^N_{i=1}\frac{p_i^2}{m_i}+\frac{p_{\epsilon}^2}{W}-(N_f+1)kT\nonumber
\end{align}
When $\sum_{i=1}^NF_i=0$, there's an conserved energy
\begin{align}
H'=H(p,r)+\frac{p_{\epsilon}^2}{2W}+\frac{p_{\xi}^2}{2Q}+LkT\xi+P_{ext}V
\end{align}
and an additional momentum conservation law
\begin{align}
\bf{P}e^{\epsilon+\xi}=K
\end{align}
where $\bf{P}=\sum_{i=1}^N$, for
\begin{align}
\frac{d}{dt}\bf{P}e^{\epsilon+\xi}&=\sum_{i=1}^N\bigg(-\frac{p_{\epsilon}}{W}p_i-\frac{p_{\xi}}{Q}p_i\bigg)+\bf{P}e^{\epsilon+\xi}\frac{Dp_{\epsilon}}{W}+\bf{P}e^{\epsilon+\xi}\frac{p_{\xi}}{Q}=0
\end{align}
The compressibility of the equations is
\begin{align}
\kappa_{Hoover}&=\sum_{i=1}^N\bigg(\frac{\partial}{\partial r_i}\cdot\dot{r}_i+\frac{\partial}{\partial p_i}\cdot\dot{p}_i\bigg)+\frac{\partial}{\partial V}\dot{V}+\frac{\partial}{\partial p_{\epsilon}}\cdot\dot{p}_{\epsilon}\nonumber\\
&=-\frac{(DN+1)p_{\xi}}{Q}+\frac{Dp_{\epsilon}}{W}\\
&=-(DN+1)\dot{\xi}+D\dot{\epsilon}\nonumber
\end{align}
We can obtain the phase space metric 
\begin{align}
\sqrt{g_{Hoover}}=\frac{1}{V}e^{DN+1}
\end{align}
Only taking the conserved energy into consideration, the partition function becomes
\begin{align}
\Omega_{T,P_{ext}}(N,E)=&\frac{e^{\beta E}}{LkT}\int dp_{\xi}e^{\beta p_{\xi}^2/2Q}\int dp_{\epsilon}e^{-\beta p_{\epsilon}^2/2W}\nonumber\\
&\int dV\frac{1}{V}e^{-\beta P_{ext}V}\int d^Np\int d^Nre^{-\beta H(p,r)}
\end{align}
Due to the presence of the $\frac{1}{V}$ factor in the volume integration, the volume distribution is incorrect. The difficulty arises from the equations of motion don't have the desired compressibility.
\subsection{The MTK Algorithm}
Martyna, Tobias and Klein introduced an algorithm which has been proved to yield a correct volume distribution.
\begin{align}
\dot{r}_i&=\frac{p_i}{m_i}+\frac{p_{\epsilon}}{W}r_i\nonumber\\ 
\dot{p}_i&=F_i-\bigg(1+\frac{D}{N_f}\bigg)\frac{p_{\epsilon}}{W}p_i-\frac{p_{\xi}}{Q}p_i\nonumber\\
\dot{V}&=\frac{dp_{\epsilon}}{W}V\nonumber\\
\dot{p_{\epsilon}}&=dV(P-P_{ext})+\frac{D}{N_f}\sum_{i=1}^{N}\frac{p_i^2}{m_i}-\frac{p_{\xi}}{Q}p_{\epsilon}\\
\dot{\xi}&=\frac{p_{\xi}}{Q}\nonumber\\
\dot{p_{\xi}}&=\sum^N_{i=1}\frac{p_i^2}{m_i}+\frac{p_{\epsilon}^2}{W}-(N_f+1)kT\nonumber
\end{align}
Compared to Hoover's algorithm, this one add a term to yield an extra$-\frac{dp_\epsilon}{W}$ in the compressibility. The $p_\epsilon$ equation has been modified to ensure the energy conservation.\\
Other thermostat with better behavior can replace the Nos\'{e}-Hoover thermostat.
\end{document}



