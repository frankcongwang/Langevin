\documentclass{article}
\usepackage{amsmath}
\usepackage{mathrsfs}
\title{Expression of Pressure }
\author{Frank}
\begin{document}
\maketitle
Since the kinetic energy is a universal term that appears in all Hamiltonian, we can introduce the configuration partition function
\begin{align}
Z(N,V,T)=\int_{D(V)}dr_1\cdots dr_Ne^{-\beta U(r_1\cdots r_N)}
\end{align}
And the pressure is
\begin{align}
P=kT\frac{\partial}{\partial V}lnQ(N.V,T)=\frac{kT}{Z(N.V,T)}\frac{\partial Z(N.V,T)}{\partial V}
\end{align}
It can be seen immediately that the volume dependence is contained in the integration limit, so that the volume differentiation cannot be easily performed. \\
So we can introduce the scaled coordinates with the definition 
\begin{align}
s_i=\frac{1}{V^{1/3}}r_i
\end{align}
\end{document}