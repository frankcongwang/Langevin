\documentclass{article}
\usepackage{amsmath}
\usepackage{mathrsfs}
\title{Legendre Transform }
\author{Frank}
\begin{document}
\maketitle
If we specify the slope $s_0=g(x_0)=f'(x_0)$ and the y-intercept $b(x_0)$ at $x_0$, then $f(x_0)$ can be uniquely determined
\begin{align}
f(x_0)=f'(x_0)x_0+b(x_0)
\end{align}
Since it is valid for all $x_o$, it can be written generally in terms of $x$ as
\begin{align}
f(x)=f'(x)x+b(x)
\end{align}
Recognizing $x=g^{-1}(s)$ and assuming $s=g(x)$ exits and is one-to-one-mapping, it's clear that the function $b(g^{-1}(s))$, given by
\begin{align}
b(g^{-1}(s))=f(g^{-1}(x))-sg^{-1}(s)
\end{align}
contains the same information as the original $f(x)$ but expressed as a function of s instead of x.\\
We call the function $\tilde{f}(s)=b(g^{-1}(s))$ the \textbf{Legendre transform} of $f(x)$.\\
$\tilde{f}(s)$ can be written compactly as
\begin{align}
\tilde{f}(s)=f(x(s))-sx(s)
\end{align}
And for a function of $n$ valuables the Legendre transform of $f$ is
\begin{align}
\tilde{f}(s_1,...,s_n)=f(x_1(s_1,...,s_n),...,x_n(s_1,...,s_n))-\sum^n_{i=1}s_ix_i(s_1,...,s_n)
\end{align}
\end{document}
